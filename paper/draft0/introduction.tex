\section{Introduction} 
\label{sec:intro}

As described by~\cite{mcfarlin2013discerning}, the exceptionally high
performance of out-of-order (OoO) processors is fundamentally due to two key
attributes: dynamism and speculation.  Lack of either feature can significantly
impact its performance. 

The key sources of energy consumption in OoO processors is data-movement and __
(back this up).

BBS is a hardware-software scheduling hybrid technique that enables significant
energy savings in the processor while maintaining near the same level of
performance as the out-of-order processor. This hybrid model is constructed
based on the notion of coarse-grain out-of-order execution where the term
coarse-grain refers to the ability of the processor to execute groups of
instructions out of program order. Each group executes its instructions in order
while multiple such groups execute their instructions out of program order.
