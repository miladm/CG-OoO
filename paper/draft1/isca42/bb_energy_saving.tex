\section{Energy Savings of BBE}
\label{sec:energy_sav}

In this section we discuss the design details of each of the energy saving
opportunities evaluated in BBE.

Discussion on the ISA structure to save energy fetching instructions. Also
evaluate the effect of adding one instruction to the instruction que. Also
discuss the ease of decoding in this process. Perhaps there is energy saving in
this as well? How much?

Branch prediction lookup overhead elimination through H instructions.

Discussion on register renaming done through H instructions only. This allows
faster instruction dispatch because while the header is renaming other
instructions in BB don't have to be unblocked. the bb will only be marked
available when its operands are all renamed. read operands are kelt in the GRF
scoreboard table and write operands stay in the bbROB entry.

Discussion on short-lived operands handled through local registers solution to
eliminated excessive register file lookup, update and commit and in turn
enabling a register structure with fewer physical registers. discuss the local
register assignment algorithm here.
\subsection{Register Allocation}
\label{sec:reg_alloc}

Describe the notion of local registers vs. global registers.

Discuss the algorithm to contruct local registers.

(a discussion on turning global registers into temporary local registers)

\subsubsection{Register File Structure}
\label{sec:reg_files}

We find that on average roughly 50\% of data communication in SPEC benchmarks
are done across def-uses that have sub-basicblock live ranges. Such registers
do not need to be register renamed if an additional register space is available
for them to temporarily hold their values for the upcoming user instruction.  As
a result, two classes of registers are defined in this architecture: local registers
and a global register. The global register is a big register file with register
renaming support. Local registers, on the other hand, are small register
file blocks with no register renaming; allocation for these registers is
done at compile time.  Global registers are designed for data communication
across basicblocks while local registers are designed for data communication
within a bsaicblock.


Discussion on cheap issue and update operations through basicblock buffer and
register scoreboards. also talk about the savings in the number of cycles.

Squasg speed improcements under the new squash model. Must discuss new register
renaming model (should I just talk about this when I discuss register renaming
        above?). Need a comparison of this model vs. the conventional squash
model. Perhaps I should show comparison against checkpointing as well.
Discussion on the design details of register renaming. This must be followed by
a comparison of the renaming performance and energy compared to the conventional
renaming method. also evaluate pipeline depth saving.
\subsection{BB Register Renaming Model}
\label{sec:bb_reg_ren}

BBE enables the design of an energy efficient register rename model with one
register aliasing table that holds both architectural and speculative register
states.

Upon a squash, we want to flush younger basic-blocks and invalidate any write
that has been done to the PR by checking the RAT. This way, the drain stage is
fully flushed out, making long-pipeline designs extremely affordable.

We still have to spend some cycles cleaning up the PR (but not nearly as must we
        would pay with RR model in OOO)

It is also much more energy efficient.


Load-store conflict reduction through static schedule of instructions.

commit benefits under bbROB: enables bulk squash, runs the chance of wasted
instructions (show how many on avg), much smaller ROB with fewer ports and much
smaller size.
\subsubsection{Basicblock Reorder Buffer (BBROB)}
\label{sec:bb_rob}

Contrary to the OoO execution model where the program order is tracked at
instruction granularity through the reorder buffer (ROB), in this design, we only
track program order at basic-block granularity. BBROB is a content addressable
memory array (CAM) with only 16 entries (~10x smaller than ROB size in OoO) that
enables issuing, committing, and squashing basic-blocks as a whole in one cycle
(per basic-block). The smaller size of BBROB compared to a ROB enables
significant energy savings.

BBROB is marked complete when all its operations are completed. Complete
operands increment the completion counter in their BBROB entry. Once the counter
reaches the expected BB size, the BB is marked complete. BBE can commit up to
one BB per cycle, allowing bulk commit using one port to the table. At commit,
    the global write operands in the BB will be marked non-speculative.

%Discussion on what is held on each BBROB entry: BBID, number of completed
%basic-block instructions, total number of instructions in the BB that need to
%complete, valid bit, mis-speculation bit, global physical register writes.

%Discussion on how basic-block entry updates the GRF upon commit through its
%global physical register writes (its GRF write register(s) go from speculative
%to architectural).

%After the decode stage, a BB entry is reserved in BBROB, and when all
%instructions in a basic-block are completed and the basic-block reaches the head
%of the BBROB, the basic-block is committed as a whole in one cycle. Upon a branch
%mis-speculation event (either branch or memory), younger basic-blocks are flushed
%in one cycle. (TODO: talk more about the squash process here)


Static schedule of instructions to leverage effective data forwarding results.
