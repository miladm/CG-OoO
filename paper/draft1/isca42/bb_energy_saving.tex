\section{Energy Savings of BBE}
\label{sec:energy_sav}

In this section we discuss the design details of each of the energy saving
opportunities evaluated in BBE.

Discussion on the ISA structure to save energy fetching instructions. Also
evaluate the effect of adding one instruction to the instruction que. Also
discuss the ease of decoding in this process. Perhaps there is energy saving in
this as well? How much?

Branch prediction lookup overhead elimination through H instructions.

Discussion on register renaming done through H instructions only. This allows
faster instruction dispatch because while the header is renaming other
instructions in BB don't have to be unblocked. the bb will only be marked
available when its operands are all renamed. read operands are kelt in the GRF
scoreboard table and write operands stay in the bbROB entry.

Discussion on short-lived operands handled through local registers solution to
eliminated excessive register file lookup, update and commit and in turn
enabling a register structure with fewer physical registers. discuss the local
register assignment algorithm here.

Discussion on cheap issue and update operations through basicblock buffer and
register scoreboards. also talk about the savings in the number of cycles.

Squasg speed improcements under the new squash model. Must discuss new register
renaming model (should I just talk about this when I discuss register renaming
        above?). Need a comparison of this model vs. the conventional squash
model. Perhaps I should show comparison against checkpointing as well.
Discussion on the design details of register renaming. This must be followed by
a comparison of the renaming performance and energy compared to the conventional
renaming method. also evaluate pipeline depth saving.
%\subsection{BB Register Renaming Model}
\label{sec:bb_reg_ren}

BBE enables the design of an energy efficient register rename model with one
register aliasing table that holds both architectural and speculative register
states.

Upon a squash, we want to flush younger basic-blocks and invalidate any write
that has been done to the PR by checking the RAT. This way, the drain stage is
fully flushed out, making long-pipeline designs extremely affordable.

We still have to spend some cycles cleaning up the PR (but not nearly as must we
        would pay with RR model in OOO)

It is also much more energy efficient.


Load-store conflict reduction through static schedule of instructions.

commit benefits under bbROB: enables bulk squash, runs the chance of wasted
instructions (show how many on avg), much smaller ROB with fewer ports and much
smaller size.

Static schedule of instructions to leverage effective data forwarding results.
