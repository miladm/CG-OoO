\section{Energy Overheads of OoO}
\label{sec:o3_overhead}

Dynamic instruction scheduling and speculative execution deliver high single
threaded performance at the expense of keeping a large amount of instruction
state data in various tables such as the reservation station, register rename,
      and load-store queue tables. Overall XX\% of the overall CPU energy is
      consumed maintaining these tables. In this section we discuss how various
      design choices in the OoO processor design lead to wasted energy, and in a
      later section we discuss solutions to address each of the problems
      outlined here.

In modern OoO processors, the branch prediction unit (BPU) must be accessed
before the corresponding fetch-group is decoded and instruction types are
identified~\cite{ref:EV8}. As a result the BPU is accessed by all instruction,
    creating excessive access traffic and mis-prediction due to aliasing.
    Identifying branch instructions ahead of time and looking up BPU for only
    branch operations is desirable as it reduces the look up traffic by XX\% and
    reduce the number of BPU to XX.

[NOT SURE IF I WILL KEEP THIS PARAGRAPH YET]
The number of architectural registers (AR) in the ISA define the number of bits
required to represent a register operand, and the number of physical registers
(PR) define the number of bits required to represent a physical register operand
after register renaming. These bits are XX\% and XX\% of the instruction binary
in x86 and ARM ISA's~\cite{}. Energy is spent in carrying these bits through
most of the pipeline stages until when they are used to access the register
file. We observe, often times, in the program control flow nearby operations use
the same AR and/or PR operands. A solution that consolidates these operands so that
they are only carried through the micro-architecture once per cluster of
neighboring instructions is favorable. An operand compression solution 
increases the instruction fetch bandwidth without adding extra ports to the cache
or maintain the same fetch bandwidth with a narrower and more energy efficient
fetch width. Also, compressed operand encoding can simplify the decoder stage as
fewer bits are decoded. 

We observe XX\% of AR's allocated in the compiler are simple DEF-USE pairs
(without further reuse), and XX\% of the time both the DEF-USE operations belong
to the same basic-block in the control flow. Allocating register rename
resources to such cases is an energy and hardware overhead that, if avoided,
          reduces register rename table and physical register file sizes; it
          also reduces the number of ports needed to these tables. 

Modern OoO processors hold 128 instructions in the instruction window. Dynamic
out-of-order instruction scheduling requires the entire table be parallel
searched for ready operations and updates to pending operation operands. As a
result, the instruction window is designed as a content addressable memory (CAM)
    array. Every instruction window lookup consumes XpJ on average while the
    simple instruction window queue used in in-order processors consumes XpJ on
    average to access only the head of the queue. In a later section we discuss
    a new design solution that enables dynamic instruction scheduling while
    eliminating the instruction window. 

% need a citation for next paragraph
Data forwarding allows significant performance improvements by saving a cycle
reading from the register file. While the OoO execution model benefits from data
forwarding, there is no concrete model to explicitly schedule nearby DEF-USE
operations. Since the information to generate DEF-USE chains is available at
compile time, a static scheduling model that exposes explicit forwarding
improves performance and reduces the energy overhead of unnecessary reads to the
register file.

% need a citation for the numbers here
Speculative execution squash penalty for modern superscalar OoO processors is
approximately 23 cycles. As the pipeline depth increases, squash penalty rises
to a point where speculative execution degrades the performance rather than
improving it. The bulk of the squash penalty is in committing instructions older
than the instruction causing the mis-speculation.  The reason why in squash
recovery, all older instruction must commit is for the program to reach a
non-speculative register file state.  Program checkpointing has been proposed to
eliminate the need for the re-order buffer (ROB) to avoid this
problem~\cite{checkpoint}.  While quite effective in reducing squash penalty
cost, program checkpoints utilizes additional tables to keep track of multiple
program state snapshots.  On average, these tables consume XX\% extra energy
compared to ROB-based designs.  An energy efficient approach that supports fast
squash recovery without paying the cost of checkpoint table structures is
discussed later in this paper. This design employs a renaming technique that
uses a single register alias table (RAT) as opposed to multiple RAT's at
different execution checkpoints.

%Squash support in this model eliminates the need for a register rename commit
%RAT only because on a squash the younger basicblocks can invalidate the faulty
%entries (talk about it the solution later and give a teaser here)

The OoO design choices discussed in this section introduce a degree of
complexity that leads to deep pipelining in recent processors. Simplifying these
complexities reduces on-chip data movement and program context tracking which in
turn leads fewer pipeline cycles at the same clock frequency.

%NO PLAN TO TALK ABOUT THIS FOR NOW:
%LSQ: fewer mis-speculations because clos-by LS's can't conflict
%We find that XX\% of instruction overhead is due to instruction scheduling 

