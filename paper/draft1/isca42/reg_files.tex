\subsubsection{Register File Structure}
\label{sec:reg_files}

We find that on average roughly XX\% of data communication in SPEC benchmarks
are done across def-uses that have sub-basic-block live ranges. Such registers
do not need to be register renamed if an additional register space is available
for them to temporarily hold their values for the upcoming user instruction.  As
a result, two classes of registers are defined in this architecture: local registers
and global registers. The global register is a 128-entry register file with register
renaming support. Local registers, on the other hand, are small register
file blocks with no register renaming; allocation for these registers is
done at compile time.  Global registers are designed for data communication
across basic-blocks while local registers are designed for data communication
within a basic-block. As shown in Section~\ref{sec:discussion}, the
required physical register file size and its number of ports is smaller than
OoO, making the global register file lookup much cheaper. 
%quantify the last sentence
