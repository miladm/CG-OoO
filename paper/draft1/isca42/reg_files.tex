\subsubsection{Register File Structure}
\label{sec:reg_files}

We find that on average roughly 50\% of data communication in SPEC benchmarks
are done across def-uses that have sub-basicblock live ranges. Such registers
do not need to be register renamed if an additional register space is available
for them to temporarily hold their values for the upcoming user instruction.  As
a result, two classes of registers are defined in this architecture: local registers
and a global register. The global register is a big register file with register
renaming support. Local registers, on the other hand, are small register
file blocks with no register renaming; allocation for these registers is
done at compile time.  Global registers are designed for data communication
across basicblocks while local registers are designed for data communication
within a bsaicblock.

A discussion of the compiler algorithm to construct local registers goes here.
