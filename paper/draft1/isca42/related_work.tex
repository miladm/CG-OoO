\section{Related Work}
\label{sec:rel_work}
In this section, we discuss the related work to Basic-block execution (BBE).

Braid~\cite{braid} focuses on static partitioning of instructions into based on
the program data-flow graph to construct instruction clusters called Braids.
Each braid runs as an independent cluster similar to how BBE dispatches
instructions. BBE follows the same line of design methodology with the focus on
evaluating energy saving opportunities of coarse-grain out-of-order scheduling.
The static instruction scheduler schedules each basic-block as whole rather than
fragmenting it into smaller pieces with the goal of minimizing dynamic bulk
scheduling.

Multiscalar~\cite{multiscalar} evaluates a multi-processing units capable of
steering coarse grain code segments that are potentially larger than a
basic-block to its functional units for computation. It uses replaced register
files in each stage to enable communicating all live registers between two
computation units. Multiscalar can be computation units are configurable to be
OoO or InO. BBE solves the same line of problem through a much more energy
efficient design.

Complexity Effective~\cite{complexity} proposes a distributed instruction
window that simplifies the wake-up logic, issue window, and the forwarding logic
design. Unlike BBE, instruction scheduling / steering in Complexity Effective is
done at instruction granularity.

ILDP~\cite{ildp} proposes a architecture for distributed processing that
consists of a hierarchical register file built for communication short lived
registers locally and long lived registers globally similar to BBE. ILDP,
          however, focuses on proposing design building blocks for
          constructing distributed processing architectures. BBE utilizes this
          design methodology to evaluate energy efficiency gains of coarse-grain
          execution.

WiDGET~\cite{widget} is a power proportional grid execution design consisting of
decoupled thread context management and a large set of simple execution units.
It has a static instruction steering protocol to effectively perform instruction
scheduling at runtime.  WiDGET is an extension of teh work by Salverda and
Zilles's\cite{fundamental} work on designing an instruction scheduling cost
model. BBE is a bulk code scheduling solution allowing new energy saving
opportunities.


TRIPS / EDGE~\cite{edge}~\cite{trips} is a high-performance, grid-processing
architecture that uses pure static instruction scheduling using hyperblocks to
map instructions to the grid of computational units. Its primary focus is
improving extracting ILP, TLP and DLP from the program rather than saving energy
efficiency. Hyperblocks use branch predication to group basic-blocks that are
connected connected together through weakly biased branches. While effective for
improving instruction parallelism, hyperblock codes lead to energy inefficient
mis-speculation recovery events that are not suitable for energy aware
technologies.

CLP~\cite{clp} addresses the processor power efficiency concern through a
configurable clustering of simple execution units to build variable-issue
computing units to exploit both thread-level parallelism and instruction level
parallelism. Core Fusion~\cite{corefusion} has a similar design mentality,
    proposing a configurable chip multiprocessor (CMP) architecture. BBE focuses
    on designing an energy efficient and complexity effective single-threaded
    processor.

iCFP~\cite{icfp} is a variant of the CFP~\cite{cfp} design for in-order
processors. It addresses the head-of-queue blocking problem in the InO processor
by building an architecture that, on every cache miss, checkpoints the program
context, steers miss-dependent instructions to a side buffer enabling
miss-independent instruction to make forward progress. CFP enables the use of
small register file and instruction window while keeping the while maintaining
the same level of performance as conventional OoO processors.  BOLT~\cite{bolt}
is a high ILP, high MLP, latency-tolerant (LT) architecture design for energy
efficient out-of-order execution. It uses a slice buffer design that utilizes a
minimal hardware resources. Unlike BBE, iCFP, CFP, and BOLT perform lightweight
{\it{dynamic}} instruction scheduling to manage available instructions and hide
LLC cache misses.

Outrider~\cite{outrider} is a high-throughput architecture that decouples a
single thread context into two parallel {\it{strands}} one for memory and
another for other operations in order to allow LLC miss latency hiding by
helping miss-independent instructions fast-forward through the execution
pipeline. Its goal is to eliminate the herd-to-scale processor structures such
as the instruction window and the OoO scheduling model. BBE is a single-threaded
design focused on hiding LLC latency miss through issuing instructions form
multiple energy efficient BB Window FIFO buffers.

Trace Processors~\cite{trace} propose a similar register hierarchy and
instruction flow design based on dynamic trace processing in the frontend. BBE
is an energy efficient solution focused on offloading majority of instruction
scheduling work to the compiler.

MorphCore~\cite{morphcore} is an in-order, out-of-order processor hybrid
designed to improve single-threaded energy efficiency. It uses {\{it{dynamic}}
instruction scheduling and core-type detection. It achieves 22\% improvement in
energy-delay product while BBE achieves 25\%.

Sun ROCK~\cite{rock} is a dynamic chip multithreading processor with support for
efficient program checkpointing. It is an out-of-order with dynamic instruction
scheduling.
