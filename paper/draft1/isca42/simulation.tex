\section{Simulation Methodology}
\label{sec:simulation}

We use our cycle accurate decoupled timing and functional simulator built on
Pin~\cite{pin}. To faithfully model speculative program execution,  we model
wrong-path code execution using Pintool Context Manipulation API. We assume 20
cycles of branch mis-speculation penalty.

For dynamic branch prediction, the $2Bc{-}gskew$ predictor designed for low
prediction aliasing is used~\cite{ref:seznec1999aliased},~\cite{ref:EV8}.
Similar to~\cite{ref:EV8}, the $e{-}gskew$ predictor consists of two global
predictors (G0, G1) and a bimodal predictor (BIM) which is also used as a
standalone predictor. A choice predictor (META) chooses between BIM and
$e{-}gskew$. Global branch history (HIST) is used to hash into G0 and G1.

To capture microarchitecture area and energy, we synthesize logical tables and
wires in the 45 nm TSMC node. Our tool evaluates multi-ported SRAM cells
designed as either content addressable memory (CAM) or random access memory
(RAM) arrays.  It computes area and energy numbers for a single read / write
access. These energy numbers are imported to our simulation framework to
accurately track the energy characteristics of each logic unit.
%more discussion on how each table energy is broken into CAM and RAM and modeled
%separately.

For evaluating system performance, we use SPCE2006 Integer benchmarks with ref
inputs~\cite{spec}. All benchmarks are fast-forwarded by one billion
instructions, warmed up for three million instructions, and evaluated for
performance and energy consumption for twenty million instructions.

Table~\ref{tab:params} and~\ref{tab:cores} provide the
architectural detail we use for the in-order, out-of-order, and basic-block
execution cores. The information provided in these tables assume a 4-wide
superscalar processor for all CPU's. Our evaluation of 2-wide and 8-wide
machines scale down / up all {\it{width}} factors by 2x.

The information provided in Table


\begin{table}[htbp]
\centering
\begin{tabular}{ l|l }
\hline
ISA					&x86\\ \hline
Technology          &45nm \\ \hline
Clock Rate          &2GHz\\ \hline
L1 Data Cache       &32kB, 8-way assoc, 4-cyc latency\\ \hline
L2 Cache size       &256KB, 8-way assoc, 12-cyc latency\\ \hline
L3 Cache size       &4MB, 8-way assoc, 20-cyc latency\\ \hline
Main Memory         &100 cyc latency\\ \hline
Branch Predict	    &2Bc-gskew, 20-cyc mis-predict penalty\\ \hline
-- BP Tables        &16Kb G0,G1,META;4Kb BIM;13b HIST\\ \hline
BTB                 &4K entries \\ \hline
\end{tabular}
\caption{Common Parameters}
\label{tab:params}
\end{table}

\begin{table}[htbp]
\centering
\begin{tabular}{ l|l }
\hline
\textbf{Core}           & \textbf{In-Order}\\ \hline
Pipeline Depth          & 9 cycles\\ \hline
Instruction Win         & 50 entries, 4 read, 4 write ports, FIFO \\ \hline
Execution Unit          & 4 general-purpose functional units \\ \hline
Register File           & 80 entries, 8 read, 4 write ports\\ \hline

\textbf{Core}           & \textbf{Out-of-Order}\\ \hline
Pipeline Depth          & 11 cycles\\ \hline
Register Rename         & 8 source, 4 destination operands \\ \hline
Reservation Stn         & 128-entry, 4 read, 4 write ports, CAM \\ \hline
Execution Unit          & 4 general-purpose functional units \\ \hline
Register File           & 256 entries, 8 read, 4 write ports\\ \hline
Re-order Buffer         & 180 entries, 4 read, 4 write ports\\ \hline
Load/Store Queue        & 64 LQ entries, 32 SQ entries, CAM \\ \hline

\textbf{Core}           & \textbf{Basic-block Execution}\\ \hline
Pipeline Depth	        & 10 cycles\\ \hline
Register Rename         & 8 source, 4 destination operands \\ \hline
BB Window               & 8 16-entry, 4 read, 4 write ports, FIFO \\ \hline
Execution Unit          & 4 general-purpose functional units \\ \hline
Global Register File    & 128 entries, 8 read, 4 write ports \\ \hline
Local Register File     & 8 entries, 1 read, 1 read/write ports \\ \hline
BB Re-order Buffer      & 8 entries, 4 read, 4 write ports \\ \hline
Load/Store Queue        & 64 LQ entries, 32 SQ entries, CAM \\ \hline
\end{tabular}
\caption{Core Parameters}
\label{tab:cores}
\end{table}
