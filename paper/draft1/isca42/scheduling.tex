\subsection{Basicblock Scheduling}
\label{sec:scheduling}

Instruction scheduling example.

Scheduling consists of:

1) ready check: find ready instructions from head of BB - what are the
dependency checks done?

2) update check: blast write operands to all BB-header buffers. 

3) discussion of available bbWindow detection and bbWidnow allocation

% This unit is broken into two main sub-units: instruction issue and instruction
% update. The instruction issue unit looks for ready instructions at the
% head of each basicblock window buffer on every cycle. The instruction udpate
% unit updates global and local register values.
% 
% Update unit discussion goes here. It talks about updating local registers and
% their ready bits, and updating basicblock headers to indicate if the global
% operands of different instructions in a basicblock are ready.
% 
% Issue unit discussion goes here. It talks about how the issue unit checks the
% corresponding local and global operands of an instruction to determine if the
% instruction is ready. Global operands availability is checked by looking up the
% BB header (where the update unit marks a valid bit for each pgysical global
% operand that is ready) and the local operands are checked by lookup
% their up valid bit in the corresponding LRF. 
% 
% Here we also discuss the energy efficiency implications of looking up the head
% of 16 basicblock buffers rather than blasting through a 120 entry instruction
% window (which is a CAM array). We also discuss how basicblock headers help
% reduce the number of check-for-ready-operands accesses to GRF by keeping ready
% information for each global register value in the header of the BB.
% %TODO you need to implement this
