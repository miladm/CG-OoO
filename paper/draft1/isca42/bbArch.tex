\documentclass[pageno]{jpaper}

%replace XXX with the submission number you are given from the ISCA submission site.
\newcommand{\iscasubmissionnumber}{XXX}

\usepackage[normalem]{ulem}
\usepackage[nocompress]{cite}
\usepackage{epsfig}
\usepackage{graphicx}
\usepackage{epstopdf}
\usepackage{ragged2e}
\usepackage{hyperref}
%\usepackage{amssym}
\usepackage{amsmath}
\usepackage{algorithm}
\usepackage{algpseudocode}
\usepackage{listings}
\usepackage{courier}

\begin{document}

\title{
Basic-block Architecture}

\date{}
\maketitle

%\thispagestyle{empty}

%%%%%%% ABSTRACT %%%%%%%
\begin{abstract}

The main factors contributing to the bulk of energy consumption in modern
out-of-order processors are the speculative logic units, dynamic instruction
scheduling modules, and on-chip data movement.  This work reduces
these energy overheads via introducing compiler-generated techniques that enable
near in-order energy consumption while maintaining near out-of-order computation
performance. Specifically, we introduce the notion of coarse-grain program
execution; in this model, program basic-blocks are speculatively run
out-of-order while instructions within each basic-block run in-order.
Basic-block execution closes up to XX\% of the energy gap and XX\% of the
performance gap between in-order and out-of-order.

\end{abstract}

%%%%%%% BODY %%%%%%%
\section{Introduction} 
\label{sec:intro}
Basic-block execution (BBE)

Discussion on single-threaded computing

Energy behavior or O3 and its shortcomings

Performance behavior of O3 and its benefits (speculation and dynamism)

Discussion of related papers that address above issues. items are: 
1) dynamism and speculation computing~\cite{dyn_specul},
2) multi-scalar
3) energy efficient computing (recent papers on this)


\begin{figure}[h]
	\centering
	\includegraphics[width=1.0\columnwidth]{fig/energy_perf_insight.pdf} 
	\caption{Energy-Performance trend of recent micro-architectures}
	\label{fig:insight}
\end{figure}


% As described by~\cite{}, the exceptionally high
% performance of out-of-order (OoO) processors is fundamentally due to two key
% attributes: dynamism and speculation.  Lack of either feature can significantly
% impact its performance. 
% 
% The key sources of energy consumption in OoO processors is data-movement and XX
% (back this up).
% 
Dynamic execution and program speculation enable significant single-threaded
performance benefits through effective latency hiding of unpredictable events
such as cache miss and control mis-speculation.  Despite their performance
advantages, these execution models produce significant energy overhead for
keeping track of instruction states and generate substantial on-chip data
traffic; this energy overhead may be acceptable when, in fact, an unpredictable
event stalls the execution flow; otherwise, a statically scheduled program can
perform at least as well as a dynamically generated schedule.

This paper makes the following contributions:
\begin{itemize}
    \item It evaluates the impact of dynamic / static hybrid instruction
    scheduling on energy efficient and high-performance computing. To do so, it
    evaluates coarse-grain execution as a tool to avoid program context tracking
    at instruction granularity.
    \item It evaluates an energy-aware set of compilation strategies including
    local register renaming, block level instruction scheduling, ahead branch
    prediction support.
    \item It provides a new front-end model in which the branch prediction unit
    is only accessed by branch instruction addresses without introducing fetch
    stall bubbles in the pipeline.
    \item It introduces coarse-grain squash with neither context checkpointing
    nor re-order buffer drain support.
    \item It evaluates the performance benefits of the register renaming
    algorithm with one register alias table as a tool for fast program squash
    support and energy efficient renaming.
    \item It provides a 10x shorter re-order buffer model that tracks program
    order at basic-block granularity, enabling bulk instruction commit and fast
    squash restart.
\end{itemize}

Section~\ref{sec:rel_work} covers the related work,
    section~\ref{sec:o3_overhead} discusses the major energy bottlenecks of the
    OoO execution, section~\ref{sec:course_grain} describes our energy
    efficiency vision behind coarse-grain execution, section~\ref{sec:code_gen}
    discusses our energy-aware compilation methodology, section~\ref{sec:uarch}
    provides the Bsaic-block execution model microarchitecture,
    section~\ref{sec:simulation} has the simulation methodology,
    section~\ref{sec:discussion} discusses the results achieved in this work,
    and section~\ref{sec:conclusion} concludes the paper.

\section{Related Work}
\label{sec:rel_work}

Have a score table of related work.

Papers to talk about:

\begin{itemize}
    \item \cite{multiscalar} multiscalar
    \item \cite{widget} widget
    \item \cite{complexity} Complexity-Effective
    \item \cite{fundamental} Salverda \& Zilles
    \item \cite{clp} CLP
    \item \cite{rock} ROCK
    \item \cite{icfp} iCFP
    \item \cite{outrider} outrider
    \item \cite{bolt} BOLT
    \item \cite{edge} EDGE
    \item \cite{cfp} CFP
    \item \cite{trace} trace processors
    \item \cite{ildp} ILDP
    \item \cite{tls} TLS
    \item \cite{braid} Braid
    \item \cite{corefusion} Core fusino
    \item \cite{complexity} complexity effective
    \item \cite{morphcore} MorphCore
    \item \cite{scalable_frontend} Scalable front-end architecture
    \ldots
\end{itemize}

\section{Energy Overheads of OoO}
\label{sec:o3_overhead}

iWindow: update and ready are CAM lookups. both take up a lot of energy

Register Renamer: renaming for short lived registers: lots of table lookups,
         large port count

Register file: lots of renaming means lots of PR's and lots of ports

BPU: lookup per fetch-width is excessive and hence prone to energy loss and
aliasing problems. can only do lookup for H instructions
to reduce port count + traffic of access.

having smaller instructions reduces the decoding overhead making the instruction
decoding energy much smaller.

Squash drain overhead. This process involves either draining the entire ROB upto
the point of flush or define tables that checkpoint program state. The latter
approach is significantly more high performance, esp. for deep pipeline
structures while being more energy consuming for keeping track of program state.
WE choose to evalaute our work against the case with only ROB. 

Squash support in this model eliminates the need for a register rename commit
RAT only because on a squash the younger basicblocks can invalidate the faulty
entries (talk about it the solution later and give a teaser here)

LSQ: fewer mis-speculations because clos-by LS's can't conflict

Pipeline stage resduction: faster BP, fast and fewer RR, faster scheduling

forwarding: arbitrary in OOO, but here it is scheduled to leverage it.


fetch overhead is completely justified because each instruction in this ISA is
shortened, making more than enough room to fetch additional instructions (i.e.
        H).
 

\section{Coarse-grain Dynamic Execution}
\label{sec:course_grain}

In this work, we introduce the notion of coarse-grain execution where every
cluster of instructions is a unit of execution that is statically scheduled to
run optimally without further dynamic scheduling by the hardware. We call it the
{\it{Phraseblock}}. Phraseblock differs from other instruction clustering
definitions such as superblocks~\cite{superblock} and
hyperblocks~\cite{hyperblock} in that its purpose is to group the set of
instructions that, ideally, can execute without an unpredictable latency
interruption. 

The simplest form of a phraseblock is the program basic-block whose energy and
performance are evaluated in this work. Similar to an instruction, the
basic-block is a single-entry, single-exit unit without any control stalls.
Memory stalls are possible within a basic-block, but with effective static
scheduling, their impact on stalling other independent instructions in the
basic-block is minimized.

\begin{figure}
	\centering
	\includegraphics[width=0.5\columnwidth]{fig/coarse_grain_sch.pdf} 
	\caption{Coarse-grain execution model}
	\label{fig:coarse_grain_sch}
\end{figure}

In the presence of unpredictable memory and control stalls, a fully static
schedule cannot deliver the same code quality as a dynamic scheduler like the
Tomosulo's algorithm~\cite{tomasulo}. To address the problem with unpredictable
long latency operations, we propose a new approach in combining static and
dynamic scheduling that leverages the ability of static scheduling in saving
energy while delivering optimal code schedules for parts of the code that need
dynamic scheduling and to leverage the ability of dynamic scheduling in hiding
the latency of unpredictable long latency events such as cache misses. To do so,
    we allow multiple speculative basic-blocks to be in-flight at the same time
    to dynamically contend for resources.  Each basic-block is statically
    scheduled and issues its instructions in-order.  As a result, when a load
    operation in a basic-block misses in the L1, it stalls but other in-flight
    basic-blocks issue instructions to hide its latency.
    Figure~\ref{fig:coarse_grain_sch} illustrates four basic-blocks placed in
    four first-in-first-out (FIFO) instruction queues, contending to schedule
    their instructions on one of the available execution units (EU).

\section{Code Generation}
\label{sec:code_gen}

In this section, we discuss the compiler passes used to construct BBE program
binary as well as the micro-architectural blocks designed to support BBE.

We run binary translation on the x86 ISA to generate the program control flow.
Once basic-blocks are identified, a number of elements in each BB must be set;
special basic-block header (\texttt{Head}) instructions must be inserted, an
energy-aware register allocation model is run, and instruction scheduling is
done to generate the optimal static schedule for each basic-block.  In this
work, a basic-block is defined to end with a \texttt{br}, \texttt{jmp},
    \texttt{call}, or \texttt{return}. It is also terminated when the the
    basic-block reaches 16 instructions. %TODO specify how rare is this

\begin{figure}
	\centering
	\includegraphics[width=1.0\columnwidth]{fig/header_ins.pdf} 
    \caption{\texttt{Head} and \texttt{Head2} instructions represent BB
        header information. \texttt{Head2} is only used for basic-blocks with
            more than three global read operands. BR ADDR represents the address
            offset of the branch operation at the end of the BB. V bits specify
            how many of the register operands are valid and whether BR ADDR
            holds a valid entry.}
	\label{fig:header_ins}
\end{figure}

Two register allocation passes are done on the compiler to generate instruction
operands. The first pass finds the register operands live beyond the boundaries
of the basic-block, and the second pass allocates registers live only within the
basic-block boundaries. The former is a global register and the latter is a
local register. The motivation to build separate register operands is to avoid
register renaming on short-lived registers and instead store them in statically
managed, small and energy efficient register files. As discussed later, we find
the energy difference between accessing a physical register file and a local
register to be about 14x. To further save energy, for cases with a global DEF
followed by one or more global USE of the same physical register, a copy of the
operand is stored locally for the USE operation(s). When more then one operand
in a basic-block reads from a global register, a \texttt{MOV loc, glb}
operation is inserted prior to the reads to bring the operation to the local
register space prior to the readers. With these assumptions, global read and
write operands within each basic-block access the global register file once.

At runtime, the decoder stage identifies the beginning of a new basic-block by
the special {\it{Header}} instruction. As shown in Figure~\ref{fig:header_ins}
Header contains 1) the instruction address of the branch instruction in the BB
(if any) and 2) the global ready architectural register read operands. BR ADDR
is used to lookup the BPU to find the next basic-block. It is stored in Header
to 1) initiate fetch of the next speculative BB early, and 2) to enable BPU
access {\it{after}} instruction decode, when Head is detected, but branch is
potentially not yet fetched so that {\it{only}} Head instructions access BPU
rather than all fetched instructions. Read register operands are removed from
their original operands and compressed into Head to 1) enable renaming bypass by
all instructions except Head, and 2) shorten the frontend pipeline depth.
\texttt{Head2} is an extension to \texttt{Head} for BB's with more than three
global operands. %TODO how often?

%TODO discuss instruction scheduling


\section{Compiler structure}
\label{sec:arch}

In this section, we discuss the compiler passes used to construct BBE program
binary as well as the micro-architectural blocks designed to support BBE.

We run binary translation on the x86 ISA to generate the program control flow.
Once basic-blocks are identified, a number of elements in each BB must be set;
special basic-block header (\texttt{Head}) instructions must be inserted, an
energy-aware register allocation model is run, and a pass of instruction
scheduling is done to generate the optimal static schedule for each basic-block.
In this work, a basic-block is defined to end with a \texttt{br}, \texttt{jmp},
   \texttt{call}, or \texttt{return}. It is also terminated when the
   the basic-block reaches 16 instructions. %TODO specify how rare is this

Two register allocation passes are done on the compiler to generate instruction
operands. The first pass finds the register operands live beyond the boundaries
of the basic-block, and the second pass allocates registers live only within the
basic-block boundaries. The former is a global register and the latter is a
local register. The motivation to build separate register operands is to avoid
register renaming on short-lived registers and instead store them in statically
managed, small and energy efficient register files. As discussed later, we find
the energy difference between accessing a physical register file and a local
register to be about 14x. To further save energy, for cases with a global DEF
followed by one or more global USE of the same physical register, a copy of the
operand is stored locally for the USE operation(s). When more then one operand
in a basic-block reads from a global register, a \texttt{MOV loc, glb}
operation is inserted prior to the reads to bring the operation to the local
register space prior to the readers. With these assumptions, global read and
write operands within each basic-block access the global register file once.

At runtime, the decoder stage identifies the beginning of a new basic-block by
the special {\it{Header}} instruction. As shown in Figure~\ref{fig:header_ins}
Header contains 1) the instruction address of the branch instruction in the BB
(if any) and 2) the global ready architectural register read operands. BR ADDR
is used to lookup the BPU to find the next basic-block. It is stored in Header
to 1) initiate fetch of the next speculative BB early, and 2) to enable BPU
access {\it{after}} instruction decode, when Head is detected, but branch is
potentially not yet fetched so that {\it{only}} Head instructions access BPU
rather than all fetched instructions. Read register operands are removed from
their original operands and compressed into Head to 1) enable renaming bypass by
all instructions except Head, and 2) shorten the frontend pipeline depth.
\texttt{Head2} is an extension to \texttt{Head} for BB's with more than three
global operands. %TODO how often?

%TODO discuss instruction scheduling

\begin{figure}
	\centering
	\includegraphics[width=1.0\columnwidth]{fig/header_ins.pdf} 
    \caption{\texttt{Head} and \texttt{Head2} instructions represent BB
        header information. \texttt{Head2} is only used for basic-blocks with
            more than three global read operands. BR ADDR represents the address
            offset of the branch operation at the end of the BB. V bits specify
            how many of the register operands are valid and whether BR ADDR
            holds a valid entry.}
	\label{fig:header_ins}
\end{figure}

\section{Microarchitecture}

%Coarse grain execution exposes energy saving opportunities in most pipeline
%stages. In this section we discuss the flow of basicblocks through different
%pipeline stages and elaborate on how energy is saved in each stage.

Figure~\ref{fig:bb_arch} shows the basic-block execution micro-architecture. The
hardware units distinguishing BBE from OoO are Basic-block Window (BB Window),
         Local Register File (LRF), Basic-block Re-order Buffer (BB-ROB), the
         Register Rename bypass line, basic-block scheduler, and the branch
         Prediction lookup line coming out from the BB Scheduler rather than
         instruction scheduler. 

provide an example of the execution here.  it should contain a code with
multiple BB's with local/global registers, header ins.  It should show flow of
BB in a 2-wide machine. provide execution schedule, update and issue register
activity. instruction issue logic information.

\subsection{CPU Frontend}
\label{sec:cpu_frontend}

discussion of register renaming and branch prediction lookup and update models.

% The processor frontend consists of the branch-prediction, fetch, decode, and
% register rename stages. Branch prediction only looks up H instructions. Fetch
% and decode stages are designed similar to existing architectures with
% configurable withs of 2, 4, 8. Contrary to the OoO model where register renaming
% happens in dispatch stage, here register renaming is done right after decode
% (explain why).  The register rename unit is only accessed by H instructions.
% Local registers do not access the register rename unit.
% 
% Given the smaller utility of register renaming, this processor can use a
% smaller set of physical registers (less area and energy per access).


\begin{figure}
	\centering
	\includegraphics[width=1.0\columnwidth]{fig/bb_architecture.pdf} 
	\caption{Basicblock execution microarchitecture}
	\label{fig:bb_arch}
\end{figure}


\begin{figure*}
	\centering
	\includegraphics[width=\textwidth]{fig/pipeline.pdf} 
	\caption{Basicblock execution pipeline}
	\label{fig:pipeline}
\end{figure*}


\subsection{CPU Backend}
\label{sec:cpu_backend}

The backend consists of several basicblock window buffers used to store
basicblock operations and the meta-data associated with each BB, local registers
for each basicblock, a global register file, execution units, basicblock reorder
buffer (BBROB), a load-store queue model, and the wake-up logic.

\subsubsection{Basicblock Window}
\label{sec:bb_window}

BB window is a small buffer, typically, with 16 entries to hold instructions. It
has a small buffer space to store the address of global read and write registers
of the basicblock operations. BB window is accessed in a FIFO manner as each basicblock is
designed to execute in-order.

This architecture has 16 basicblock window buffers allowing for concurrent
execution of instructions from multiple basicblocks speculatively. Each cycle,
the head of each buffer is accessed to issue ready instructions.  Instruction
scheduler issues instructions from older basicblocks first.

Figure XX shows the average number of in-flight basicblocks for SPEC2006
benchmark.

\subsubsection{Register File Structure}
\label{sec:reg_files}

We find that on average roughly 50\% of data communication in SPEC benchmarks
are done across def-uses that have sub-basicblock live ranges. Such registers
do not need to be register renamed if an additional register space is available
for them to temporarily hold their values for the upcoming user instruction.  As
a result, two classes of registers are defined in this architecture: local registers
and a global register. The global register is a big register file with register
renaming support. Local registers, on the other hand, are small register
file blocks with no register renaming; allocation for these registers is
done at compile time.  Global registers are designed for data communication
across basicblocks while local registers are designed for data communication
within a bsaicblock.

A discussion of the compiler algorithm to construct local registers goes here.

\subsubsection{Basicblock Reorder Buffer (BBROB)}
\label{sec:bb_rob}

Contrary to the OoO execution model where the program order is tracked at
instruction granularity through the reorder buffer (ROB), in this design, we only
track program order at basic-block granularity. BBROB is a content addressable
memory array (CAM) with only 16 entries (~10x smaller than ROB size in OoO) that
enables issuing, committing, and squashing basic-blocks as a whole in one cycle
(per basic-block). The smaller size of BBROB compared to a ROB enables
significant energy savings.

BBROB is marked complete when all its operations are completed. Complete
operands increment the completion counter in their BBROB entry. Once the counter
reaches the expected BB size, the BB is marked complete. BBE can commit up to
one BB per cycle, allowing bulk commit using one port to the table. At commit,
    the global write operands in the BB will be marked non-speculative.

%Discussion on what is held on each BBROB entry: BBID, number of completed
%basic-block instructions, total number of instructions in the BB that need to
%complete, valid bit, mis-speculation bit, global physical register writes.

%Discussion on how basic-block entry updates the GRF upon commit through its
%global physical register writes (its GRF write register(s) go from speculative
%to architectural).

%After the decode stage, a BB entry is reserved in BBROB, and when all
%instructions in a basic-block are completed and the basic-block reaches the head
%of the BBROB, the basic-block is committed as a whole in one cycle. Upon a branch
%mis-speculation event (either branch or memory), younger basic-blocks are flushed
%in one cycle. (TODO: talk more about the squash process here)



%\subsection{Basicblock Scheduling}
\label{sec:scheduling}

This unit is broken into two main sub-units: instruction issue and instruction
update. The instruction issue unit looks for ready instructions at the
head of each basicblock window buffer on every cycle. The instruction udpate
unit updates global and local register values.

Update unit discussion goes here. It talks about updating local registers and
their ready bits, and updating basicblock headers to indicate if the global
operands of different instructions in a basicblock are ready.

Issue unit discussion goes here. It talks about how the issue unit checks the
corresponding local and global operands of an instruction to determine if the
instruction is ready. Global operands availability is checked by looking up the
BB header (where the update unit marks a valid bit for each pgysical global
operand that is ready) and the local operands are checked by lookup
their up valid bit in the corresponding LRF. 

Here we also discuss the energy efficiency implications of looking up the head
of 16 basicblock buffers rather than blasting through a 120 entry instruction
window (which is a CAM array). We also discuss how basicblock headers help
reduce the number of check-for-ready-operands accesses to GRF by keeping ready
information for each global register value in the header of the BB.
%TODO you need to implement this


\subsection{Squash Handling}
\label{sec:speculation}

In BBE, squash events are handled at basic-block granularity. To avoid wasting
{\it{useful}} instructions executed in a mis-speculated BB, the compiler uses
profiling information to separate weakly biased branches form the rest of the BB
into a single-instruction BB. Less than 0.01\% of useful instructions are wasted
in BBE.

The average number of BB's 



Mis-speculation handling is coarse grain, making it fast and energy efficient.
Discussion on number of wasted instructions.
Discussion on the changes in memory mis-speculation model.

%There are two kinds of speculation: memory and control.
%
%Control speculation: apart from the oldest basicblock in flight, all other
%in-flight basicblocks are speculatively fetched and executed. The key elements
%of speculation are: branch prediction, squash handling. Branch prediction is
%done energy efficiently and more accurately as only H instructions access BP.
%Squash handling is done coarse-grain meaning there is a chance that we lose
%useful instructions (SPEC shows that the number useful instructions flushed is
%small - I will add a figure to back it here). We also talk about how the number
%of cycles to ramp down from a squash is shorter in BB  OoO.
%
%Memory mis-prediction is less likely in this model as each basicblock buffer
%runs in-order (no chance of mis-speculation between load-stores within a
%basicblock). We also say that mis-speculation handling is the same for
%both memory and branch.


%TODO discuss how large basicblocks are partitioned.
%TODO Talk about exception handling here.

%\section{Energy Savings of BBE}
\label{sec:energy_sav}

In this section we discuss the design details of each of the energy saving
opportunities evaluated in BBE.

Discussion on the ISA structure to save energy fetching instructions. Also
evaluate the effect of adding one instruction to the instruction que. Also
discuss the ease of decoding in this process. Perhaps there is energy saving in
this as well? How much?

Branch prediction lookup overhead elimination through H instructions.

Discussion on register renaming done through H instructions only. This allows
faster instruction dispatch because while the header is renaming other
instructions in BB don't have to be unblocked. the bb will only be marked
available when its operands are all renamed. read operands are kelt in the GRF
scoreboard table and write operands stay in the bbROB entry.

Discussion on short-lived operands handled through local registers solution to
eliminated excessive register file lookup, update and commit and in turn
enabling a register structure with fewer physical registers. discuss the local
register assignment algorithm here.
\subsection{Register Allocation}
\label{sec:reg_alloc}

Describe the notion of local registers vs. global registers.

Discuss the algorithm to contruct local registers.

(a discussion on turning global registers into temporary local registers)

\subsubsection{Register File Structure}
\label{sec:reg_files}

We find that on average roughly 50\% of data communication in SPEC benchmarks
are done across def-uses that have sub-basicblock live ranges. Such registers
do not need to be register renamed if an additional register space is available
for them to temporarily hold their values for the upcoming user instruction.  As
a result, two classes of registers are defined in this architecture: local registers
and a global register. The global register is a big register file with register
renaming support. Local registers, on the other hand, are small register
file blocks with no register renaming; allocation for these registers is
done at compile time.  Global registers are designed for data communication
across basicblocks while local registers are designed for data communication
within a bsaicblock.

A discussion of the compiler algorithm to construct local registers goes here.


Discussion on cheap issue and update operations through basicblock buffer and
register scoreboards. also talk about the savings in the number of cycles.

Squasg speed improcements under the new squash model. Must discuss new register
renaming model (should I just talk about this when I discuss register renaming
        above?). Need a comparison of this model vs. the conventional squash
model. Perhaps I should show comparison against checkpointing as well.
Discussion on the design details of register renaming. This must be followed by
a comparison of the renaming performance and energy compared to the conventional
renaming method. also evaluate pipeline depth saving.
\section{BB Register Renaming Model}
\label{sec:bb_reg_ren}

Upon a squash, we want to flush younger basic-blocks and invalidate any write
that has been done to the PR by checking the RAT. This way, the drain stage is
fully flushed out, making long-pipeline designs extremely affordable.

We still have to spend some cycles cleaning up the PR (but not nearly as must we
        would pay with RR model in OOO)

It is also much more energy efficient.


Load-store conflict reduction through static schedule of instructions.

commit benefits under bbROB: enables bulk squash, runs the chance of wasted
instructions (show how many on avg), much smaller ROB with fewer ports and much
smaller size.
\subsubsection{Basicblock Reorder Buffer (BBROB)}
\label{sec:bb_rob}

Contrary to the OoO execution model where the program order is tracked at
instruction granularity through the reorder buffer (ROB), in this design, we only
track program order at basic-block granularity. BBROB is a content addressable
memory array (CAM) with only 16 entries (~10x smaller than ROB size in OoO) that
enables issuing, committing, and squashing basic-blocks as a whole in one cycle
(per basic-block). The smaller size of BBROB compared to a ROB enables
significant energy savings.

BBROB is marked complete when all its operations are completed. Complete
operands increment the completion counter in their BBROB entry. Once the counter
reaches the expected BB size, the BB is marked complete. BBE can commit up to
one BB per cycle, allowing bulk commit using one port to the table. At commit,
    the global write operands in the BB will be marked non-speculative.

%Discussion on what is held on each BBROB entry: BBID, number of completed
%basic-block instructions, total number of instructions in the BB that need to
%complete, valid bit, mis-speculation bit, global physical register writes.

%Discussion on how basic-block entry updates the GRF upon commit through its
%global physical register writes (its GRF write register(s) go from speculative
%to architectural).

%After the decode stage, a BB entry is reserved in BBROB, and when all
%instructions in a basic-block are completed and the basic-block reaches the head
%of the BBROB, the basic-block is committed as a whole in one cycle. Upon a branch
%mis-speculation event (either branch or memory), younger basic-blocks are flushed
%in one cycle. (TODO: talk more about the squash process here)


Static schedule of instructions to leverage effective data forwarding results.

\section{Simulation Framework}
\label{sec:simulation}

Discussion on the simulator parameters, use of Pintool, pinpoints, wrong-path
execution support, energy model integration.

Discussion on the benchmark

Simulator framework: pin, wrong-path, energy, speculation, function vs. timing
simulation, pin-points?

\section{Results Discussion}
\label{sec:discussion}

Show cool cool results :).

%\section{outline}
\label{sec:outline}

In this paper I would like to point out the following achievements we have
made:

\begin{itemize} 
    \item energy efficient and high performance computing need for dynamism and
speculation (INTRO)
    \item compiler design
    \item architecture design
    \item pipeline energy efficiency opportunities
    \item ?
    \item results
    \item recovery time, energy, performance, register file behavior, LSQ,
wakeup logic information, having multiple issue elements from each BB
    \item simulation framework
    \item discussion
\end{itemize}


%%%%%%% CITATIONS %%%%%%%
\bstctlcite{bstctl:etal, bstctl:nodash, bstctl:simpurl}
\bibliographystyle{IEEEtranS}
\bibliography{references}

\end{document}

